\subsection*{Apresentação}

O presente projeto visa desenvolver o aluno na prática de programação orientada a objetos usando a biblioteca Qt.

O projeto consiste em três programas de computador que trabalham em conjunto para simular um sistema simples de aquisição e supervisão de dados usando comunicação T\+C\+P/\+IP em uma rede local.

Em suma, os três módulos devem ser capazes de realizar as seguintes operações\+:


\begin{DoxyItemize}
\item O {\bfseries servidor} deve esperar conexões T\+CP destinadas à porta 1234, e responder ao cliente conforme os comandos que este enviar.
\item O {\bfseries cliente produtor} de dados deve ser capaz de se conectar a uma máquina executando o servidor na porta 1234 e enviar, usando comandos específicos, marcações de data/hora juntamente com uma informação em ponto flutuante para ser gravada.
\item O {\bfseries cliente supervisor} de dados deve ser capaz de se conectar a uma máquina executando o servidor na porta 1234 e recuperar, usando comandos específicos, a lista dos clientes produtores de dados, bem como listagens de dados produzidos por um destes clientes produtores.
\end{DoxyItemize}

O aluno deverá desenvolver apenas os {\bfseries cliente produtor} e o {\bfseries cliente supervisor}. O módulo {\bfseries servidor} já está pronto e não precisa ser trabalhado.

\subsection*{O módulo servidor}

O módulo servidor implementa o que se chama de servidor T\+CP. Em outras palavras, esse programa é capaz de escutar a rede local e aguardar por conexões remotas destinadas à porta T\+C\+P/1234.

Em redes T\+C\+P/\+IP, o protocolo de comunicação T\+CP permite a criação de um circuito virtual, um canal de comunicação que pode ser usado para enviar e receber sequências de bytes pela Internet. O canal é fechado apenas quando a conexão é interrompida.

Para se abrir uma conexão com uma máquina que executa um determinado serviço usando o protocolo T\+CP é necessário que se conheça seu endereço IP (ou nome) e uma {\itshape porta} onde o serviço será provido. Quando a conexão é aberta para um novo cliente, inicia-\/se um {\itshape socket} de comunicação, identificado, entre outras coisas, pela combinação I\+P+porta. Cada {\itshape socket} possui um número único que pode ser usado para distinguir entre as várias conexões que podem chegar à mesma porta. Isso é comum em máquinas que provêem serviços a vários clientes.

Máquinas que aguardam conexões comumente chamadas de {\bfseries servidores}. O servidor implementado neste projeto {\itshape escuta} a porta {\bfseries 1234}. Uma vez que um cliente se conecte a esta, as tarefas que o servidor irá executar dependerão de mensagens enviadas pelo cliente. Para cada mensagem, uma tarefa diferente é executada. É dessa maneira que os vários serviços na Internet funcionam.

O servidor do projeto não necessita de modificações para funcionar. Basta abrir o projeto no Qt\+Creator, compilar e executar o código. O servidor é capaz de interpretar mensagens em texto simples que lhe forem enviadas. As mensagens aceitas pelo servidor formam o que se chama de {\bfseries protocolo de aplicação} para este serviço. Três mensagens são suportadas nesse protocolo\+:


\begin{DoxyCode}
list
get NUMERO\_IP N\_AMOSTRAS
set DATA\_E\_HORA\_EM\_MS DADO
\end{DoxyCode}


O comando $\ast$$\ast$\+\_\+list\+\_\+$\ast$$\ast$ retorna a lista de máquinas cujos dados produzidos encontram-\/se armazenados no servidor. ex\+: 
\begin{DoxyCode}
$ telnet 127.0.0.1 1234
list
127.0.0.1
\end{DoxyCode}


O comando $\ast$$\ast$\+\_\+get\+\_\+$\ast$$\ast$ precisa que se forneça também o número IP do {\bfseries cliente produtor} que se deseja recuperar o conjunto de dados produzidos. ex\+:


\begin{DoxyCode}
$ telnet 127.0.0.1 1234
get 127.0.0.1 1
1496658174409 34
\end{DoxyCode}


O comando $\ast$$\ast$\+\_\+set\+\_\+$\ast$$\ast$ precisa que se forneça uma combinação D\+A\+TA e H\+O\+RA, bem como o dado que se deseja armazenar no servidor. ex\+:


\begin{DoxyCode}
$ telnet 127.0.0.1 1234
set 127.0.0.1
set 1496658174409 34
\end{DoxyCode}


Todos os comandos devem ser enviados na forma de literais. O indicador de data/hora é um long que armazena a quantidade de milisegundos passados desde a data 1/1/1970. O servidor, uma vez que receba essas sequências de literais, separa-\/as conforme a quantidade de espaços presentes e armazena os dados associados em uma estrutura local criada para esse fim.

Em se tratando de um projeto meramente acadêmico, pouco controle de erro é realizado nessa versão inicial.

Quando o servidor é executado, uma mensagem é exibida na aba {\bfseries Application Output} do Qt\+Creator mostrando o IP local do servidor, como ilustram as linhas a seguir\+:


\begin{DoxyCode}
server started at:
"192.168.1.106"
\end{DoxyCode}


\subsection*{O módulo cliente produtor de dados}

O cliente produtor de dados deve ser capaz de se conectar a um servidor em execução e enviar comandos {\bfseries set} para este servidor. Espera-\/se que esse módulo seja capaz de simular um processo de coleta e envio de dados para o servidor em intervalos periódicos.

A construção do cliente deve possibilitar o envio de dados (possivelmente aleatórios) para o servidor em intervalos regulares de tempo. Para isso, seu usuário deverá ser capaz de realizar as seguintes operações no módulo produtor\+:


\begin{DoxyItemize}
\item Indicar o IP do servidor ao qual se deseja conectar
\item Indicar a faixa de valores (min-\/max) que poderão ser enviados ao servidor
\item Indicar o intervalo de tempo entre o envio de dois dados consecutivos
\item Realizar a conexão ao pressionar um botão
\item Iniciar a transmissão de dados ao pressionar um botão
\item Finalizar a transmissão de dados ao pressionar um botão
\item Exibir os dados enviados em uma janela de texto
\end{DoxyItemize}

Um exemplo de interface para o módulo produtor de dados é mostrado na figura que segue\+:



\subsection*{Consumidor de dados}

O consumidor de dados assume um papel mais complexo que o produtor. O produtor deve ser capaz de se conectar a um servidor indicado, recuperar os endereços das máquinas que produziram dados e traçar um gráfico data-\/hora x valor de uma sequência de dados fornecida pelo servidor.

O usuário do módulo consumidor de dados, o usuário deverá ter à sua disposição as seguintes funcionalidades\+:


\begin{DoxyItemize}
\item Indicar o endereço IP do servidor de dados ao qual o cliente deverá se conectar.
\item Iniciar a conexão com o servidor ao se pressionar um botão.
\item Uma vez conectado, o usuário deverá ter a disposição uma lista das máquinas que produziram dados no servidor. Isso pode ser implementado usando um {\bfseries listwidget}.
\item No {\bfseries listwidget} o usuário deverá ser capaz de selecionar a máquina cujos dados deseja visualizar.
\item Com uma máquina selecionada, dois botões deverão ficar disponíveis para o usuário\+: um para começar a receber os dados e outro para parar de receber.
\item Deverá ficar disponível a possibilidade de modificar o intervalo de tempo de captura entre dois conjuntos de dados consecutivos, ou seja, entre o envio de dois comandos \+\_\+get 
\end{DoxyItemize}

Para cada recepção de dados, a ferramenta deverá traçar o gráfico tempo x valor do conjunto de dados lido do servidor. Algumas formas de traçado podem ser consideradas\+:


\begin{DoxyItemize}
\item Traçar todo intervalo de valores lido do servidor.
\item Traçar um subintervalo dos valores lidos (ex\+: desenhar apenas as últimas 30 amostras obtidas). Isso pode interessante para evitar que o gráfico fique sobrecarregado.
\end{DoxyItemize}

Um exemplo de interface para o módulo consumidor de dados é mostrado na figura que segue\+:

 